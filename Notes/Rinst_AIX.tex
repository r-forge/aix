\documentclass[a4paper]{report}

%% my packages
\usepackage{hyperref}
%%\usepackage{amsmath}
%%\usepackage{amsfonts}
\usepackage[utf8]{inputenc}
%%\usepackage[round]{natbib}
\usepackage{a4wide}

%%\usepackage{Sweave}
%%\usepackage{enumerate}

%
\newcommand{\strong}[1]{{\normalfont\fontseries{b}\selectfont #1}}
\newcommand{\class}[1]{\mbox{\textsf{#1}}}
\newcommand{\func}[1]{\mbox{\texttt{#1()}}}
\newcommand{\code}[1]{\mbox{\texttt{#1}}}
\newcommand{\pkg}[1]{\strong{#1}}
%\newcommand{\samp}[1]{`\mbox{\texttt{#1}}'}
\newcommand{\proglang}[1]{\textsf{#1}}
\newcommand{\set}[1]{\mathcal{#1}}

%% almost as usual
\author{Stefan Theu\ss{}l}
%% Working Title
\title{Update to \proglang{R} Installation and Administration Manual}

\begin{document}

\appendix
\section{AIX}

\proglang{R} has been built successfully on AIX 6.1 on a `powerpc'
target, more specifically on a platform running a Power6 processor, by
a group around Kurt Hornik at the Wirtschaftsuniversit\"at in
Vienna. We no longer support AIX prior to 4.2, and configure will
throw an error on such systems. Since \proglang{R}-2.9 it is possible
to configure without problems using the additional
argument~\code{--enable-R-shlib}.

A small part of the software needed to build \proglang{R} and/or
install packages 
is available directly from the AIX Installation DVDs, e.g., \code{Java6},
\code{X11}, or \code{perl}. Further open source software (OSS) is
packaged for AIX in so-called .rpm files and available from
the ``AIX Toolbox for Linux Applications'' of
IBM~\url{http://www-03.ibm.com/systems/power/software/aix/linux/} as
well as from \url{http://www.oss4aix.org/download}. The latter website
typically offers more recent versions of the available OSS. All needed
tools and libraries downloaded from those repositories (e.g.,
\code{gcc}, \code{gfortran}, \code{libreadline}, \code{make}, etc.)
are typically installed to \code{/opt/freeware} and hence the
executables are found in \code{/opt/freeware/bin}, so if using those
tools ensure this is in your path.

Like on other Unix systems you will need GNU \code{libiconv} as the
AIX version of \code{iconv} is not sufficiently powerful.
Additionally, for proper unicode compatibility we recommend to install
the corresponding package from the ICU project
(\url{http://www.icu-project.org/download/}). The project
offers pre-compiled binaries for various platforms which in case of
AIX can be installed via unpacking the tarball to the root file
system. 

For full \LaTeX{} support we installed the \TeX{} Live DVD distribution which
we obtained from \url{http://www.tug.org/texlive/}. It is recommended
to update the distribution using the \code{tlmgr} update
manager.

For 64-bit builds of \proglang{R} supporting tcl/tk it is necessary to
install tcl/tk from sources as pre-compiled binaries supply only 32-bit
shared objects.

The recent testing has been done under AIX 6.1 using
compilers from both, the GNU Compiler Collection (version 4.2.4) which is
available from one of the above OSS repositories, and the IBM C/C++ (XL
C/C++ 10.01) as well as FORTRAN (XL Fortran 12.01) compilers
(\url{http://www14.software.ibm.com/webapp/download/byproduct.jsp#X}).

To compile for a 64-bit `powerpc' target (running a Power6 CPU) with
\code{gcc} 4 we used:

\code{CC ="gcc -maix64 -pthread"}\\
\code{CXX="g++ -maix64 -pthread"}\\
\code{FC="gfortran -maix64 -pthread"}\\
\code{F77="gfortran -maix64 -pthread"}\\
\code{CFLAGS="-O2 -g -mcpu=power6"}\\
\code{FFLAGS="-O2 -g -mcpu=power6"}\\
\code{FCFLAGS="-O2 -g -mcpu=power6"}\\

For the IBM XL compilers we used:

\code{CC=xlc}\\
\code{CXX=xlc++}\\
\code{FC=xlf}\\
\code{F77=xlf}\\
\code{CFLAGS="-qarch=auto -qcache=auto -qtune=auto -O3 -qstrict -ma"}\\
\code{FFLAGS="-qarch=auto -qcache=auto -qtune=auto -O3 -qstrict"}\\
\code{FCFLAGS="-qarch=auto -qcache=auto -qtune=auto -O3 -qstrict"}\\
\code{CXXFLAGS="-qarch=auto -qcache=auto -qtune=auto -O3 -qstrict"}\\

It is important to note for the IBM compilers that the decision
for producing 32-bit or 64-bit code is done by setting the
\code{OBJECT\_MODE} environment variable appropriately (recommended) or
use an additional compiler flag (\code{-q32} or \code{-q64}). By
default the IBM XL compilers produce 32 bit code, i.e., if you want to
build \proglang{R} to support 64-bit you need to export
\code{OBJECT\_MODE=64} to your environment or, alternatively, use the
\code{-q64} compiler option.

\subsection{Additional Notes}

For a faster run of the \code{configure} script export the environment
variable to something other than the Korn shell (\code{ksh}),
e.g. provided that you have installed \code{bash} you should export
\code{CONFIG\_SHELL=/usr/bin/bash}. Note: \code{/bin/sh} may refer to
\code{ksh} on your system. 


We recommend to use GNU \code{make} which can be forced to be used on
your Unix system by setting \code{MAKE=/opt/freeware/bin/make} in your
environment. 


Software compiled from sources is typically installed to
\code{/usr/local}, so do not forget to add \code{/usr/local/bin} to
your \code{PATH} environment variable.

Further installation instructions to setup a proper
\proglang{R} development environment can be found in the ``R on AIX''
project on R-Forge~(\url{http://r-forge.r-project.org/projects/aix/}).

\end{document}